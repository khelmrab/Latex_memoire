\chapter{Etude de l'existant}
En se basant sur plusieurs sites et tableaux comparatifs, il en est ressorti une liste de critères, assez complète, sur les DAMs actuellement disponibles sur le marché, en plus des 10 critères de base d'un DAM.

\section{Lists des critéres Caractéristiques d'un DAM :}
\begin{enumerate}
\item Gestion des droits par utilisateur ou groupes
\item Accès distant
\item Gestion des droits par fichier 	
\item Historique des utilisations	
\item API
\item Intégration des applications
\item La gestion d'actifs
\item Gestion de présence
\item Sauvegarde automatique
\item Notifications automatiques
\item Rappels automatiques
\item Barcode Scanning
\item L'intelligence d'entreprise
\item Gestion du calendrier
\item Chat
\item Notifications par email
\item Prévision
\item La géolocalisation
\item Suivi de l'historique
\item Contrôle de l'inventaire
\item Suivi de l'inventaire
\item Gestion des factures
\item Factures
\item Paiements électroniques
\item Suivi des dépenses
\item Intégration comptable
\item Gestion de la comptabilité
\item Base de connaissances
\item Portail client
\item Tableau de bord d'activité
\item Enregistrement d'activité
\item Gestion des activités
\item Surveillance de l'activité
\item Suivi de l'activité
\item Rapports ad hoc
\item Audit
\item Outils de collaboration
\item Espace de travail collaboratif
\item Gestion de la communication
\item Gestion de la conformité
\item Intégration CRM
\item Les champs personnalisés
\item Formulaires personnalisés
\item Base de données client
\item Historique des clients
\item Marquage personnalisable
\item Champs personnalisables
\item Rapport personnalisable
\item Modèle personnalisable
\item Importation des données
\item Exportation des données
\item Interface de glisser-déposer
\item Intégration de courrier électronique
\item T.P.V. (terminal point de vente) 
\item Contrôle de processus
\item Suivi de projet
\item Données en temps réel
\item Surveillance en temps réel
\item Notifications en temps réel
\item Rapports en temps réel
\item Mises à jour en temps réel
\item La gestion des dossiers
\item Rapports et statistiques
\item Rapports de vente
\item Fonctionnalité de recherche
\item Stockage sécurisé des données
\item Intégration des médias sociaux
\item État d'avancement des Rapports 
\item Planification des tâches
\item Suivi des tâches
\item Gestion des modèles (template) 
\item Emploi du temps
\item Fonctionnalités de suivi
\item Contrôles d'accès utilisateur
\item Widgets
\item Gestion du flux de travail
\item Téléchargement Par FTP	
\item Plug in pour applications tierces	
\item Formats de transcodage (vidéo)	
\item Récupération automatique de métadonnées	
\item Plan de classement	
\item Thésaurus	(C'est un langage contrôlé utilisé pour l'indexation de documents et la recherche de ressources documentaires )
\item Format des notices documentaires	
\item Gestion des métadonnées
\item Contrôle de version 
\item Taggage	
\item  Gestion de contenu
\item  Module de modification des images	
\item Module de découpage des vidéos	
\item Taggage	
\item Annotation	
\item  Gestion des droits associés aux images et vidéos	
\item  Accès en fonction de zones géographiques	
\item  Tatouage numérique (watermark)	
\item Publication	
\item Publication automatisée	
\item Connecteurs avec logiciel de Ged, d’ECM	
\item  Recherche en texte intégral sur les métadonnées	
\item  Recherche en texte intégral sur les annotations	
\item  Recherche Booléenne sur les données textuelles	
\item  Recherche Par facettes ou catégories
\item Recherche dans les images	
\item Recherche dans les vidéos	
\item Enregistrer les résultats	des recherches 
\item Possibilité de partager les résultats	de la recherche
\item Visualisation en vignettes		
\item Edition d'image 
\item Lecture des vidéos en streaming
\item Sélection par panier	
\item Commande et paiement en ligne	
\item Formation 
\item Documentation
\item Reconnaisance faciale
\item Relier des ressources(même robe de couleurs différentes)
\item Gestion fin de droits
\item Gestion d'archivage
\item Possibilité de visualisation sans login
\item Statistiques
\item Gestion sous-titres vidéos...
\item Application pour Android  
\item Application pour  iPhone-iPad
\item Stockage des documents  (word, pdf)
\item Déploiement basé sur le cloud (SaaS)
\item Déploiement sur place (On-permise)
\item Modèle de tarification (Essai gratuit,Freemium ,Licence ponctuelle,Open source,Abonnement)

\end{enumerate}

\newpage
\section{Analyse de l'existant}
On constate que les éditeurs de solutions DAM disponibles sur le marché  proposent d’autres fonctionnalités que les fonctionnalités de base afin d'avantager leurs offres et visé différents secteurs. 

Il se trouve que certains critères comme : la facturation, le paiement en ligne, l'intégration d'un mot du CRM (customer relationship management) ... etc; visent plus le secteur du e-commerce que celui du sport.


Étant donné que le domaine sportif privilégie plus l'aspect visuel (image/vidéo), on peut donc en déduire que les fonctionnalités en lien avec l'édition ou le stockage de documents (texte, pdf, tableur etc...) ne sont pas pertinentes dans notre cas.


J'ai soigneusement procédé au filtrage des fonctionnalités, que peut proposer un module de digital asset management standard, à partir de la liste générique des critères, pour que celles-ci soient les plus adaptées pour une utilisation dans le domaine du sport.

\subsection{Liste des critères pertinantes pour un DAM dédié au sport}


\begin{enumerate}
\item Gestion des droits par utilisateur ou groupes
\item Gestion des droits par fichier 
\item  Gestion des droits associés aux images et vidéos	
\item recherche par facettes (Utilisez des attributs comme les métadonnées, le type de fichier ou les mots clés pour trouver rapidement des fichiers)
\item Notifications en temps réel
\item Gestion des métadonnées	
\item Taggage	
\item Annotation	
\item  Tatouage numérique (watermark)	
\item  Gestion des contenus	
\item Contrôle de version 
\item Stockage sécurisé des données
\item Edition d'image 
\item Edition de vidéo 
\item Visualisation en vignettes	
\item Lecture des vidéos en streaming
\item Sélection par panier
\item Données en temps réel
\item Multi-Platformes
\item TéléchargementPar FTP	
\item Gestion du calendrier
\item Déploiement basé sur le cloud (SaaS)
\item Déploiement sur place (On-permise)
\item Publication	
\item Widgets
\item Importation des données
\item exportation des données
\item Application pour Android  
\item Application pour  iPhone-iPad
\item Reconnaisance faciale
\item Relier des ressources (même robe de couleurs différentes)
\newline
\end{enumerate}

\subsection{Tableau comparatif}
Voici un tableau comparatif des DAM's générique les plus utilisé par le domaine du sport (Table 2.1).
\begin{sidewaystable}[ht]
\begin{center}
\begin{tabular}{|c|c|c|c|c|c|c|c|c|c|c|}
\hline   
Critères /Nom DAMs & Keepeek & Bynder & Brandfolder & Nuxeo &  WebDAM &  Contently \\

\hline
Gestion des droits & OUI & OUI & OUI & OUI & OUI & OUI  \\
\hline	
Gestion des droits par fichier  & OUI & OUI &OUI & OUI & OUI & OUI  \\
\hline
 Recherche par facettes & OUI & OUI & OUI & OUI & OUI & OUI  \\
\hline
 La recherche avancée & OUI & OUI & OUI & OUI & OUI & OUI \\
 \hline
Notifications en temps réel & OUI & OUI & OUI &OUI & OUI & OUI \\
\hline
Gestion des métadonnées&  OUI & OUI & OUI& OUI &OUI & OUI  \\
\hline
 Taggage	 & NON & OUI & OUI & OUI &OUI &OUI  \\
\hline
 Annotation	& NON & OUI & OUI & OUI &OUI & OUI  \\
\hline
Tatouage numérique	  & NON & OUI & OUI & OUI & OUI & OUI \\
\hline
 Gestion des contenus	& OUI & OUI & OUI & OUI & OUI& OUI  \\
\hline
Contrôle de version & NON & OUI &NON & OUI & OUI & NON \\
\hline
Stockage sécurisé des données  & OUI & OUI & OUI & OUI & OUI & OUI\\
\hline
Edition d'image & OUI& OUI & OUI & OUI &OUI & OUI  \\
\hline
 Edition de vidéo  & OUI & OUI & OUI &OUI& OUI &OUI \\
\hline
 Vignettes	& OUI & OUI & OUI & OUI &OUI & OUI  \\
\hline
Lecture vidéos en streaming & OUI & OUI & OUI & OUI & OUI& OUI  \\
\hline
Sélection par panier& OUI & OUI &OUI & OUI & OUI & OUI  \\
\hline
Données en temps réel& OUI  & OUI & OUI &OUI & OUI & OUI  \\
\hline
 La géolocalisation & OUI & OUI & OUI & OUI & OUI & NON \\
\hline
TéléchargementPar FTP	& OUI & OUI & OUI & OUI & OUI & OUI  \\
\hline
Gestion du calendrier& NON & OUI & NON & NON &NON& OUI  \\
\hline
 Déploiement basé sur le cloud & OUI & OUI & OUI & OUI & OUI & OUI  \\
\hline
 Déploiement sur place& OUI & NON & OUI & OUI & NON & OUI  \\

\hline
 Publication& OUI & OUI & OUI & OUI & OUI& OUI  \\
\hline
Widgets & NON & OUI & OUI & OUI & NON & NON  \\
\hline

Relier des ressources & NON & OUI & OUI & OUI & OUI & OUI  \\
\hline
Reconnaisance faciale& OUI & NON & NON & NON & NON & NON  \\
\hline
Importation des données  &OUI & OUI & OUI & OUI & OUI & OUI  \\
\hline
Exportation des données& OUI& OUI & OUI & OUI & OUI & OUI  \\
\hline
Application pour  iPhone-iPad & OUI &NON & OUI & OUI &OUI & NON  \\
\hline
Application pour Android& NON & NON & OUI &OUI& NON &NON  \\
\hline
\end{tabular}
\caption {Tableau comparatif des DAM's générique les plus utilisé par le domaine du sport}
\end{center}
\end{sidewaystable}



\chapter{Solution proposée}
Il existe plusieurs éditeurs de DAM sur le marché, qui apportent des logiciels
génériques \textbf{très strandars} .

Cependant ! Il n’y a aucun fournisseur qui propose une solution orientée vers un domaine \textbf{précis}, qui permettrait  une visibilité par métier des fonctionnalités de ce DAM,  et cela rend la tâche du «choix d’un DAM en entreprise plus difficile , objet d'une réelle problématique d'entreprise reprise dans cette étude » , surtout pour les entreprises orientée vers la gestion de la performance , dont le processus achat en général et des solutions de gestion précisément est une fonction très stratégique à celle ci. 

\section{Proposition 1} 

La solution consiste à la mise en place d’un site Web d’aide à la décision des
entreprises avec un contenu relatif au domaine de l’événementiel sportif impliquant un filtre avec plusieurs critères , afin de faire un choix parmi les dam existants , et faciliter ainsi le choix,  l'achat et l'usage stratégiques de cette solution ciblée en entreprise 

\subsubsection{Ce site permettra de réaliser les opérations suivantes : }{} 

\begin{enumerate}
\item Trier selon les fonctionnalités que vous souhaitez.
\item Support chat 
\end{enumerate}
 
 La valeur ajoutée de ce site consiste dans le fait que les DAM sont déja triés
suivant l’analyse de chapitre : Etude de l’existant.

La confiance dans la solution proposée par ce site web est la clé pour qu’une proposition de celui ci soit acceptée, d'ou l'importance de la pertinence des critères et du filtre de ce site web , et la connaissance approfondie de ces sociétés d’événementiel sportif tendant vers la performance de leurs processus de management , notamment les services support comme le  management  des donnés  

Et c’est aussi pourquoi il est important de le développer le plus tôt possible au profit de ce profil de client.

Avec un premier filtre qui va permettre au client d’avoir des solutions ciblées dans
le contenu de l’événementiel sportif,  et le mettre ainsi plus en confiance pour faire un choix stratégique d'achat et d'usage de celle solution de DAM.
 

\section{Proposition 2}
Proposer un \textbf{DAM Spécialisée dans l’événementiel sportif}   qui rassemble les bénéfices intrinsèques liés à l’utilisation d’un DAM, la spécialisation dans le secteur de événementiel  du sport et des fonctionnalités spécifiques aux clients du secteur. Voici quelques fonctionalites :

\subsubsection {Technologie spécifique au sport}{}
\begin{enumerate}
\item Modélisation ontologique du sport (théorie des graphes). 
\item Indexation et organisation automatiques des contenus 
\item Arborescence des albums de compétitions synchronisée avec les calendriers sportif.
\item Capture et détection des relations sémantiques sportives entre les objets(images)
\item Enrichissement automatique par données contextuelles liées au sport cf. algorithme d’inférence 
\item Renforcement automatique de la visibilité des images de sport (seo)
\item Moteur de recherche sémantique avec détection des concepts sportifs 
\end{enumerate}












